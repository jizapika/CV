\documentclass[letterpaper,10pt]{article}

\usepackage{makecell}
\usepackage[link=off]{phonenumbers}

\usepackage{latexsym}
\usepackage[empty]{fullpage}
\usepackage{titlesec}
\usepackage{marvosym}
\usepackage[usenames,dvipsnames]{color}
\usepackage{verbatim}
\usepackage{enumitem}
\usepackage[pdftex]{hyperref}
\usepackage{fancyhdr}

\usepackage{xcolor}% http://ctan.org/pkg/xcolor
\usepackage{hyperref}% http://ctan.org/pkg/hyperref

%%% Работа с русским языком
\usepackage{cmap}					% поиск в PDF
\usepackage{mathtext} 				% русские буквы в формулах
\usepackage[T2A]{fontenc}			% кодировка
\usepackage[utf8]{inputenc}			% кодировка исходного текста
\usepackage[english,russian]{babel}	% локализация и переносы
\usepackage{indentfirst}
\frenchspacing

\pagestyle{fancy}
\fancyhf{} % clear all header and footer fields
\fancyfoot{}
\renewcommand{\headrulewidth}{0pt}
\renewcommand{\footrulewidth}{0pt}
\usepackage[margin=0.3in]{geometry}
% Adjust margins
\addtolength{\oddsidemargin}{-0.0in}
\addtolength{\evensidemargin}{-0.0in}
\addtolength{\textwidth}{0in}
\addtolength{\topmargin}{20pt}
\addtolength{\textheight}{0.0in}

\urlstyle{same}

\hypersetup{
    colorlinks=true,
    linkcolor=blue!50!red,
    linkbordercolor=red,
    urlcolor=blue!70!black
}

\raggedbottom
\raggedright
\setlength{\tabcolsep}{0in}

% Sections formatting
\titleformat{\section}{
    \vspace{-10pt}\scshape\raggedright\large
}{}{0em}{}[\color{black}\titlerule \vspace{-7pt}]

%-------------------------
% Custom commands
\def \ifempty#1{\def\temp{#1} \ifx\temp\empty }

\newcommand{\resumeItem}[2]{
    \item\small{
        \ifempty{#1}#2\else\textbf{#1}{: #2 \vspace{-2pt}}\fi
    }
}

\usepackage[dvipsnames]{xcolor}
\definecolor{mygray}{gray}{0}
\usepackage{fancybox}

\usepackage{lmodern}
\usepackage{tikz}

% Style definition
\tikzset{rndblock/.style={rounded corners,rectangle,draw,outer sep=0pt}}

% Command Definition
% 1 optional to customize the aspect, 2 mandatory: text to be framed
\newcommand{\tframed}[2][]{\tikz[baseline=(h.base)]\node[rndblock,#1] (h) {\color{black}{#2}};}

\newcommand*{\mystrut}{\rule[-0.2\baselineskip]{0pt}{0.8\baselineskip}}
\newcommand{\skill}[1]{\tframed[lightgray]{\mystrut#1}}


\newcommand{\resumeSubheading}[4]{
    \vspace{-1pt}\item
    \begin{tabular*}{0.97\textwidth}{l@{\extracolsep{\fill}}r}
        \textbf{#1}       & \textcolor{mygray}{#2}                 \\
        \textit{\small#3} & \textcolor{mygray}{\textit{\small #4}} \\
    \end{tabular*}\vspace{-5pt}
}

\newcommand{\resumeExpSubheading}[5]{
    \vspace{-1pt}\item
    \begin{tabular*}{0.97\textwidth}{l@{\extracolsep{\fill}}r}
        \textbf{#1}       & \textcolor{mygray}{#2}                 \\
        \textit{\small#3} & \textcolor{mygray}{\textit{\small #4}} \\
        {\scriptsize#5}
    \end{tabular*}\vspace{4pt}
}

\newcommand{\resumeProjSubheading}[4]{
    \vspace{-1pt}\item
    \begin{tabular*}{0.97\textwidth}{l@{\extracolsep{\fill}}r}
        \textbf{#1}      & \textcolor{mygray}{#2}                 \\
        \scriptsize {#3} & \textcolor{mygray}{\textit{\small #4}} \\
    \end{tabular*}\vspace{4pt}
}

\newcommand{\resumeSubItem}[2]{\resumeItem{#1}{#2}\vspace{-4pt}}

\renewcommand{\labelitemii}{$\circ$}

\newcommand{\resumeSubHeadingListStart}{\begin{itemize}[leftmargin=*]}
\newcommand{\resumeSubHeadingListEnd}{\end{itemize}}
\newcommand{\resumeItemListStart}{\begin{itemize}[leftmargin=0.2in]}
\newcommand{\resumeItemListEnd}{\end{itemize}\vspace{-5pt}}

\usepackage{changepage}
\newcommand{\resumeDesc}[1]{\begin{adjustwidth}{5pt}{0pt}\vspace{-2pt}{\small{#1}}\end{adjustwidth}}


\begin{document}

\begin{tabular*}{\textwidth}{l@{\extracolsep{\fill}}r}
\textbf{\Large Лев Чечулин, Java-разработчик} & Email : lyovoch@bk.ru \\
Github: \href{https://github.com/jizapika}{jizapika} & Мобильный телефон : +7\hspace{0.5ex}999\hspace{0.5ex}232\hspace{0.5ex}32\hspace{0.5ex}70 \\
\end{tabular*}

\section{Образование}
\resumeSubHeadingListStart
\resumeSubheading
{Национальный исследовательский университет ИТМО}
{Санкт-Петербург, Россия}
{Информационные системы и технологии}
{Сент. 2020 - Авг. 2024 (настоящее время)}
\resumeSubHeadingListEnd

\section{Опыт работы}
\resumeSubHeadingListStart

\resumeExpSubheading
{{Лаборатория Института дизайна и урбанистики Университета ИТМО}}{Санкт-Петербург, Россия}
{Разработчик на Java}{Мар. 2024 - Май 2024}
{\skill{Java 17} \skill{Gradle} \skill{Github Actions} \skill{CLI} \skill{JTS API} \skill{GeoTools API} \skill{JUnit 5} \skill{Mockito}}
\resumeDesc{
\begin{itemize}
\item Написание библиотеки для автоматизации процесса планировки застройки на неровных участках с помощью алгоритмов для выравнивания рельефа, основанных на JTS и GeoTools.
\item Покрытие кода тестами с помощью JUnit 5 и Mockito
\item Внедрение кода в систему генерации инфраструктуры.
\end{itemize}}

\resumeExpSubheading
{\href{https://yandex.ru/company/}{Яндекс}}{Санкт-Петербург, Россия}
{Стажер-разработчик на Java, \href{https://www.kinopoisk.ru/}{Кинопоиск} MWP}{Авг. 2023 - Дек. 2023}
{\skill{Java 17} \skill{Spring Boot} \skill{Spring Data JPA} \skill{WebClient vs RestTemplate} \skill{GraphQL API}}
\resumeDesc{
\begin{itemize}
\item Поддержка и добавление функциональности GraphQL API для Кинопоиска
\item Разработка коммунального асинхронного клиента для доступа к новому API другой службы с использованием WebClient,\\
Spring Security и Spring Cache.
\item Создание метрики для событий с использованием MeterRegistry и добавление оповещения о метрике.
\end{itemize}}

\resumeSubHeadingListEnd

\section{Проекты}
\resumeSubHeadingListStart

\resumeProjSubheading
{\href{https://github.com/jizapika/CarShop}{Автомобильный магазин}}{}
{\skill{Java} \skill{Spring} \skill{postgres} \skill{JPA}}{Дек. 2022}
\resumeDesc{
Изучение фреймворка Spring Boot. Работа с базой данных (postgres) с использованием JPA. \\
Приложение содержит авторизацию и регистрацию продавцов автомобилей. \\
Также предоставляет API для поиска автомобилей по цвету.
}

\resumeProjSubheading
{\href{https://github.com/is-tech-y24-1/jizapika/tree/master/2laba}{Генератор клиентов}}{}
{\skill{C\#} \skill{Java} \skill{Roslyn} \skill{LINQ}}{Сент. 2022}
\resumeDesc{
Для сервера на Java генерируется клиент на C\# с использованием Roslyn. \\
Реализован парсер модели и контроллеров в синтаксическое дерево.\\
Затем на его основе создаётся клиент на C#.
}
\resumeProjSubheading
{\href{https://github.com/is-tech-y24-1/jizapika/tree/lab-3}{Исправитель и анализатор кода}}{}
{\skill{C\#} \skill{Roslyn} \skill{LINQ}}{Май 2022}
\resumeDesc{
Анализатор находит методы, название которых начинается с префикса Try, и выдаёт предупреждение, если метод возвращает значение не типа bool. После подтверждения исправления, исправитель кода изменяет сигнатуру метода так, чтобы он возвращал bool и дополнительный выходной параметр. Разработка велась с использованием парсера Roslyn и анализа абстрактных синтаксических деревьев.
}

\resumeProjSubheading
{\href{https://github.com/is-oop-y24/jizapika/tree/master/Reports}{Система отчётности}}{}
{\skill{C\#}\skill{Swagger}}{Март 2022}
\resumeDesc{
Реализована серверная часть многоуровневого приложения, предоставляющего API для работы с системой отчётности в условиях командной разработки. Для тестирования функционала использовался Swagger. В проекте реализован уровень доступа к данным с использованием Entity Framework.
}

\resumeProjSubheading
{\href{https://github.com/is-oop-y24/jizapika/tree/master/Banks}{Банковская система}}{}
{\skill{C\#}}{Дек. 2021}
\resumeDesc{
Разработана банковская система с пользовательским интерфейсом. Центральный банк контролирует все транзакции. В каждом банке созданы счета и клиенты.
}

\resumeSubHeadingListEnd

\section{Достижения}
\resumeSubHeadingListStart

\resumeProjSubheading
{{Курс по Java от Тинькофф. Весной 2023.}}{}
{\skill{Java}\skill{Spring}\skill{Maven}\skill{JUnit5/Mockito}\skill{ORM/JDBC}\skill{Liquibase}\skill{JMS/Kafka}\skill{Docker}\skill{CI/CD}\skill{OpenAPI Specification/Swagger}}
{Март - Май 2023}
\resumeDesc{}

\resumeProjSubheading
{{Олимпиадное программирование}}{}
{\skill{C++}\skill{Алгоритмы и структуры данных}}
{2016 - 2020}
\resumeDesc{
{\href{https://neerc.ifmo.ru/archive/2020/northern/standings.html}
{ICPC (45-е место)}},
{\href{https://codeforces.com/profile/jizapika}
{Codeforces (рейтинг 2100)}}
}

\resumeProjSubheading
{{Региональные олимпиады}}{}{}{}
\resumeDesc{
Математика (2014-2020), Физика (2016-2018), Программирование (2017-2020)
}

\resumeSubHeadingListEnd

\section{Программистские навыки}
\resumeSubHeadingListStart
\item{
\textbf{Языки}{: C++, C\#, Java, Python}
}\vspace{-7pt}
\item{
\textbf{Технологии}{: SQL, Git, Bash, Maven, Docker, Kafka }
}
\vspace{-7pt}
\item{
\textbf{Знания}{: Алгоритмы, Структуры данных, Операционные системы, Базы данных, Объектно-ориентированное программирование, Машинное обучение, Веб-программирование, UML}
}
\vspace{-7pt}
\item{
\textbf{Уровень английского}{: B1}
}
\resumeSubHeadingListEnd

\end{document}